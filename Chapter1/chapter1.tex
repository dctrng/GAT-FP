\chapter{Giới thiệu}
\label{chap:Giới thiệu}


% -------------------------------------------------------------------
% Emotion Recognition in Conversation (ERC)
% -------------------------------------------------------------------

\section{Nhận Diện Cảm Xúc Trong Hội Thoại (Emotion Recognition in Conversation - ERC)}
\label{sec:Nhận Diện Cảm Xúc Trong Hội Thoại (Emotion Recognition in Conversation - ERC)}

Cảm xúc là một khía cạnh quan trọng trong giao tiếp hàng ngày của con người. 
Vì vậy, trong lĩnh vực xử lí ngôn ngữ tự nhiên, nhận diện cảm xúc là một mảng nghiên cứu càng ngày càng được quan tâm. 
Công nghệ Nhận diện cảm xúc trong hội thoại (ERC) có vai trò xác định trạng thái cảm xúc của người nói trong một cuộc hội thoại. 

ERC có nhiều ứng dụng tiềm năng như hỗ trợ đàm thoại, phân tích cho các thử nghiệm pháp lý và dịch vụ y tế điện tử, v.v.
ERC có rất nhiều tiềm năng khai thác dữ liệu từ các mạng xã hội nổi tiếng trên thế giới như Facebook, Twitter, Youtube, Reddit etc. 
ERC cũng là một thành phần quan trọng để xây dựng các tương tác máy tính tự nhiên như con người và có thể trả lời một cuộc đối thoại có tính cảm xúc.


% ----------------------------------------------------------
% Multimodal ERC:
% -------------------------------------------------------------------

\section{Multimodal ERC}
\label{sec:Multimodal ERC}

Nhận thấy các nghiên cứu về ERC tập trung vào loại dữ liệu văn bản, *trích bài báo MMGCN* đã đề xuất kết hợp nhiều kiểu dữ liệu và ERC, từ đó công bố MMGCN. 

\section{Graph-based Multimodal ERC}
\label{chap:Graph-based Multimodal ERC}







