\chapter{Phương Thức}
\label{chap:Phương Thức}

Trong chương này, chúng tôi sẽ trình bày phương thức chuẩn bị dữ liệu hội thoại, 

\section{Chuẩn Bị Dữ Liệu}

Chúng ta xác định một cuộc trò chuyện $C=\{ \left(u_{i}, y_{i}\right)\}_{i=1}^{L}$, trong đó $L$ là số lượng các câu nói trong cuộc trò chuyện, $u_{i}$ là câu nói thứ $i^{th}$ trong $C$ và $y_{i}$ là nhãn đúng của $u_{i}$. 
Ở đây, $y_{i} \in \left\{1,2,\cdots, c\right\}$ và $c$ là tổng số nhãn. 
Mỗi câu nói $u_{i}$ của người nói $p_{s(u_i)}$, trong đó hàm $s(\cdot)$ ánh xạ chỉ số của câu nói vào người nói tương ứng. For each utterance $u_{i}$, we extract multimodal features $x_{i}=\{{x}_{i}^m\}_{m\in\{a, l, v\}}$. 
Ở đây, $x_{i}^a \in \mathbb{R}^{d_a}$, $x_{i}^l \in \mathbb{R}^{d_l}$ and $x_{i}^v \in \mathbb{R}^{d_v}$ are the utterance-level features of acoustic, lexical and visual modalities, respectively. 
And $\{d_{m}\}_{m\in\{a, l, v\}}$ is the feature dimension of each modality.



